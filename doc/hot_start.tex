\documentclass[12pt,a4paper]{article}

\newcommand{\abs}[1]{\left\lvert#1\right\rvert}
\newcommand{\norm}[1]{\left\lVert#1\right\rVert}
\newcommand{\minimize}[1]{\mathop{\mbox{minimize}}_{#1} \ \ }
\newcommand{\maximize}[1]{\mathop{\mbox{maximize}}_{#1} \ \ }
\newcommand{\lex}[1]{\mbox{lex}\mathop{\mbox{minimize}}_{#1} \ \ }
\newcommand{\st}[0]{\mbox{subject to} \ \ }
\newcommand{\argmin}[1]{\mbox{arg} \min_{#1} \ }

\usepackage{graphicx,color,psfrag}
\usepackage{tabularx,colortbl}
\usepackage{amssymb}
\usepackage{amsfonts}
\usepackage{epsfig}
\usepackage[sans]{dsfont} 
\usepackage{bbm} 
\usepackage{amsmath,bm}
\usepackage{verbatim}  
\usepackage{color}     
\usepackage{subfigure} 
\usepackage{hyperref}  
\usepackage{scalefnt}
\usepackage{mathtools}

\usepackage[ruled,vlined]{algorithm2e}

\usepackage{fancyvrb} % use verbatim in align

\usepackage{accents}
\newcommand{\ubar}[1]{\underaccent{\bar}{#1}}

\usepackage{pifont}

\usepackage[top=2cm, bottom=2cm, left=2cm, right=2cm]{geometry}

\begin{document}

\SaveVerb{DEFAULT_OBJECTIVE}|DEFAULT_OBJECTIVE|
\SaveVerb{SIMPLE_BOUNDS_OBJECTIVE}|SIMPLE_BOUNDS_OBJECTIVE|
\SaveVerb{CTR_INACTIVE}|CTR_INACTIVE|
\SaveVerb{CTR_ACTIVE_EQ}|CTR_ACTIVE_EQ|
\SaveVerb{CTR_ACTIVE_UB}|CTR_ACTIVE_UB|
\SaveVerb{CTR_ACTIVE_LB}|CTR_ACTIVE_LB|
\SaveVerb{type}|type|
\SaveVerb{true}|true|
\SaveVerb{false}|false|
\SaveVerb{determine_ctr_type}|determine_ctr_type|
\SaveVerb{modify_type_inactive_enabled}|modify_type_inactive_enabled|
\SaveVerb{modify_type_active_enabled}|modify_type_active_enabled|
\SaveVerb{x_guess_is_specified}|x_guess_is_specified|
\SaveVerb{modify_x_guess_enabled}|modify_x_guess_enabled|
\SaveVerb{update_x_guess}|update_x_guess|
\SaveVerb{lexlse}|lexlse|
\SaveVerb{use_phase1_v0}|use_phase1_v0|
\SaveVerb{phase1}|phase1()|
\SaveVerb{phase1_v0}|phase1_v0()|
\SaveVerb{verifyWorkingSet}|verifyWorkingSet()|
\SaveVerb{normalIteration}|normalIteration|
\SaveVerb{set_min_init_ctr_violation}|set_min_init_ctr_violation|

\section{Hot-start related input (a summary)}

$\mathcal{W}^{\mathit{guess}}$ will denote a user guess only for the active {\bf inequality}
constraints. $\mathcal{W}^{(0)}$ is the actual initial working set which is used to solve
\UseVerb{lexlse}($\mathcal{W}^{(0)}$). $\mathcal{W}^{(0)}$ contains all equality constraints (which are
detected internally) and may contain active inequality constraints. $x^{\mathit{guess}}$ denotes the
initial guess for $x$ specified by the user, while $x^{(0)}$ is the actual initial point used in the
algorithm (they need not be the same).

\begin{itemize}
\item {\color{blue}$\{\}$}

  Equality constraints are detected internally and included in $\mathcal{W}^{(0)}$.

\item {\color{blue}$\{\mathcal{W}^{\mathit{guess}}\}$}

  $\mathcal{W}^{(0)} \leftarrow \mathcal{W}^{\mathit{guess}}$. Equality constraints are detected
  internally and included in $\mathcal{W}^{(0)}$. 

\item {\color{blue}$\{\mathcal{W}^{\mathit{guess}}, x^{\mathit{guess}}\}$} + flags (or when
  $\mathcal{W}^{\mathit{guess}} = \{\}$, one could simply use {\color{blue}$\{x^{\mathit{guess}}\}$})

  \begin{itemize}
  \item[\ding{237}] {\bf IF}
    
    \begin{itemize}
    \item \UseVerb{modify_x_guess_enabled} = \UseVerb{false}
    \item \UseVerb{modify_type_inactive_enabled} = \UseVerb{false} 
    \item \UseVerb{modify_type_active_enabled} = \UseVerb{false}
    \end{itemize}
    %
    $\mathcal{W}^{(0)} \leftarrow \mathcal{W}^{\mathit{guess}}$. Equality constraints are detected
    internally and included in $\mathcal{W}^{(0)}$. 

  \item[\ding{237}] {\bf ELSE IF}

    \begin{itemize}
    \item \UseVerb{modify_x_guess_enabled} = \UseVerb{false}
    \item {\color{blue}\UseVerb{modify_type_inactive_enabled} = \UseVerb{true}}
    \item \UseVerb{modify_type_active_enabled} = \UseVerb{false}
    \end{itemize}
    % 
    $\mathcal{W}^{(0)} \leftarrow \mathcal{W}^{\mathit{guess}}$. Additional inequality constraints
    might be activated as well. Equality constraints are detected internally and included in
    $\mathcal{W}^{(0)}$.

  \item[\ding{237}] {\bf ELSE IF}

    \begin{itemize}
    \item \UseVerb{modify_x_guess_enabled} = \UseVerb{false}
    \item \UseVerb{modify_type_inactive_enabled} = \UseVerb{false} 
    \item {\color{blue}\UseVerb{modify_type_active_enabled} = \UseVerb{true}}
    \end{itemize}
    % 
    $\mathcal{W}^{(0)} \leftarrow \mathcal{W}^{\mathit{guess}}$. Some inequality constraints
    might be dropped. Equality constraints are detected internally and included in
    $\mathcal{W}^{(0)}$.

  \item[\ding{237}] {\bf ELSE IF}

    \begin{itemize}
    \item {\color{blue}\UseVerb{modify_x_guess_enabled} = \UseVerb{true}}
    \item \UseVerb{modify_type_inactive_enabled} = \UseVerb{false} 
    \item \UseVerb{modify_type_active_enabled} = \UseVerb{false}
    \end{itemize}
    %
    $\mathcal{W}^{(0)} \leftarrow \mathcal{W}^{\mathit{guess}}$. Equality constraints are detected
    internally and included in $\mathcal{W}^{(0)}$. $x^{\mathit{guess}}$ could be modified in order
    to satisfy simple bounds (associated with the highest priority objective). This modification is
    performed with the aim of ensuring $v_i^{(0)} = 0$ corresponding to simple bounds.

  \item[\ding{237}] {\bf ELSE}

    A more complex behavior (see below).

  \end{itemize}

\item {\color{blue}$\{\mathcal{W}^{(0)}, x^{(0)}, v^{(0)}\}$}

  Advanced initialization (see Section~\ref{sec.advanced_initialization}).

\end{itemize}

\clearpage

\section{Types \& flags}

\subsection{Types of objectives}

\begin{enumerate}
\item \UseVerb{DEFAULT_OBJECTIVE}: consists of general constraints
%
  \begin{align} \label{eq.ctr}
    \ubar{b}_i \leq a_i^Tx -v_i \leq \bar{b}_i.
  \end{align}

\item \UseVerb{SIMPLE_BOUNDS_OBJECTIVE}: consists of simple bounds (only for highest priority objective)
%
  \begin{align} \label{eq.simple_bounds}
    \ubar{b}_i \leq x_i - v_i \leq \bar{b}_i.
  \end{align}

\end{enumerate}

\subsection{Types of constraints}

\begin{enumerate}
\item \UseVerb{CTR_INACTIVE}

  \begin{itemize}
  \item ({\bf as solver output}) The $i$-th constraint is inactive at the solution. 
  \item ({\bf as user input}) Specifies that the $i$-th constraint should not be included in
    the initial working set, unless
    %
    \begin{itemize}
    \item[\ding{237}] $\ubar{b}_i = \bar{b}_i$ (the $i$-th constraint is automatically modified to
      \UseVerb{CTR_ACTIVE_EQ}).
    \item[\ding{237}] \UseVerb{modify_type_inactive_enabled} = \UseVerb{true} (see Algorithm~\ref{alg.type}).
    \end{itemize}
  \end{itemize}

\item \UseVerb{CTR_ACTIVE_EQ}

  \begin{itemize}
  \item ({\bf as solver output}) The $i$-th constraint is $a_i^Tx - v_i = \ubar{b}_i = \bar{b}_i$.
  \item ({\bf as user input}) Equality constraints are detected internally in the solver and
    a user input \UseVerb{CTR_ACTIVE_EQ} is disregarded. When a constraint is set to be of type
    \UseVerb{CTR_ACTIVE_EQ} it is included in the working set (and can never be deactivated).
  \end{itemize}

\item \UseVerb{CTR_ACTIVE_LB}

  \begin{itemize}
  \item ({\bf as solver output}) Lower bound is active ($a_i^Tx^{\star} - v_i^{\star} = \ubar{b}_i$) at the solution.
  \item ({\bf as user input}) Specifies that the $i$-th constraint should be included in the
    initial working set as $a_i^Tx - v_i = \ubar{b}_i$, unless
    \begin{itemize}
    \item[\ding{237}] $\ubar{b}_i = \bar{b}_i$ (the $i$-th constraint is automatically modified to
      \UseVerb{CTR_ACTIVE_EQ}).
    \item[\ding{237}] \UseVerb{modify_type_active_enabled} = \UseVerb{true} (see Algorithm~\ref{alg.type}).
    \end{itemize}
  \end{itemize}

\item \UseVerb{CTR_ACTIVE_UB}

  \begin{itemize}
  \item ({\bf as solver output}) Upper bound is active ($a_i^Tx^{\star} - v_i^{\star} = \bar{b}_i$) at the solution.
  \item ({\bf as user input}) Specifies that the $i$-th constraint should be included in the
    initial working set as $a_i^Tx - v_i = \bar{b}_i$, unless
    \begin{itemize}
    \item[\ding{237}] $\ubar{b}_i = \bar{b}_i$ (the $i$-th constraint is automatically modified to
      \UseVerb{CTR_ACTIVE_EQ}).
    \item[\ding{237}] \UseVerb{modify_type_active_enabled} = \UseVerb{true} (see Algorithm~\ref{alg.type}).
    \end{itemize}
  \end{itemize}

\end{enumerate}

\subsection{Flags}

In this section it is assumed that $\ubar{b}_i \neq \bar{b}_i$ (recall that the type of a constraint
with $\ubar{b}_i = \bar{b}_i$ is automatically set to \UseVerb{CTR_ACTIVE_EQ}). The flags below are related
to specifying the initial guess.

\begin{itemize}

\item \UseVerb{modify_type_inactive_enabled}

  \begin{itemize}
  \item[\ding{237}] \UseVerb{false} ({\bf default}): constraints with (user specified) type \UseVerb{CTR_INACTIVE} are not included in $\mathcal{W}^{(0)}$.
  \item[\ding{237}] \UseVerb{true}: enables the solver to include additional inequality constraints to
    $\mathcal{W}^{\mathit{guess}}$, that is, some constraints with (user specified) type
    \UseVerb{CTR_INACTIVE} maybe be activated (see Algorithm~\ref{alg.type}).
  \end{itemize}

\item \UseVerb{modify_type_active_enabled}

  \begin{itemize}
  \item[\ding{237}] \UseVerb{false} ({\bf default}): constraints with (user specified) type \UseVerb{CTR_ACTIVE_LB} and \\ \UseVerb{CTR_ACTIVE_UB} are included in $\mathcal{W}^{(0)}$.
  \item[\ding{237}] \UseVerb{true}: enables the solver to modify the type of constraints in $\mathcal{W}^{\mathit{guess}}$ (or to deactivate them) (see Algorithm~\ref{alg.type}).
  \end{itemize}

\item \UseVerb{modify_x_guess_enabled}

  \begin{itemize}
  \item[\ding{237}] \UseVerb{false} ({\bf default}): $x^{\mathit{guess}}$ cannot be modified by the solver.
  \item[\ding{237}] \UseVerb{true}: enables the solver to modify $x^{\mathit{guess}}$ only for satisfying simple
    bounds in the objective with highest priority (see Algorithm~\ref{alg.x_guess}).
  \end{itemize} 

\item \UseVerb{x_guess_is_specified}

  \begin{itemize}
  \item[\ding{237}] \UseVerb{false} ({\bf default}): $x^{\mathit{guess}}$ is not specified.

  \item[\ding{237}] \UseVerb{true}: $x^{\mathit{guess}}$ is specified (by the user).

  \end{itemize} 

  \noindent When $x^{\mathit{guess}}$ has not been specified, a user input \UseVerb{CTR_INACTIVE},
  \UseVerb{CTR_ACTIVE_LB} or \UseVerb{CTR_ACTIVE_UB} is not modified in the solver (provided that
  $\ubar{b}_i \neq \bar{b}_i$). Note that the flag \UseVerb{x_guess_is_specified} is not directly set
  by the user.

\item \UseVerb{set_min_init_ctr_violation}

  \begin{itemize}
  \item[\ding{237}] \UseVerb{false} ({\bf default}) (see Sections~\ref{sec.v0_inactive}).

  \item[\ding{237}] \UseVerb{true}: set the smallest possible initial constraint violation (for inactive constraints).
  \end{itemize} 

\item \UseVerb{use_phase1_v0}

  \begin{itemize}
  \item[\ding{237}] \UseVerb{false} ({\bf default}) during the initialization use function \UseVerb{phase1} (see Sections~\ref{sec.phase1}).

  \item[\ding{237}] \UseVerb{true}: during the initialization use function \UseVerb{phase1_v0} (see Sections~\ref{sec.phase1_v0}).
  \end{itemize} 
  
\end{itemize}

\clearpage

\section{Initialization of the active-set method}

Let $\mathcal{W}^{(k)}$ be the working set at the $k$-th iteration. The solution of
\UseVerb{lexlse}($\mathcal{W}^{(k)}$) will be denoted by $(x^{\star(k)},v^{\star(k)})$, where for
$i\not\in \mathcal{W}^{(k)}$, $v_i^{\star(k)} = 0$.

\subsection{Phase 1 (default option)} \label{sec.phase1}

\begin{itemize}
\item Form $\mathcal{W}^{(0)}$ using Algorithm~\ref{alg.W0} (this could possibly modify $x^{\mathit{guess}}$).
\item Compute $(x^{\star(0)},v^{\star(0)})$ by solving \UseVerb{lexlse}($\mathcal{W}^{(0)}$).
\item Determine $x^{(0)}$
  %
    \begin{align*}
      x^{(0)} \leftarrow \left\{ 
      \begin{array}{l l}
        x^{\mathit{guess}} & \quad \text{if} \,\,\,\, \UseVerb{x_guess_is_specified} \\
        x^{\star(0)} & \quad \text{otherwise}.
      \end{array} \right.
    \end{align*}

\item Determine $v_i^{(0)}$
%
    \begin{align*}
      v_i^{(0)} \leftarrow \left\{ 
      \begin{array}{l l}
        \eqref{eq.v0_active} & \quad \text{if} \,\,\,\, i\in\mathcal{W}^{(0)} \\
        \eqref{eq.v0_inactive} & \quad \text{if} \,\,\,\, i\not\in\mathcal{W}^{(0)}.
      \end{array} \right.
    \end{align*}
    %
    If $x^{\mathit{guess}}$ is not initialized, one could use directly $v_i^{(0)} \leftarrow v_i^{\star(0)}$
    for $i\in\mathcal{W}^{(0)}$ instead of~\eqref{eq.v0_active}.

\item Determine $(\Delta x^{(0)}, \Delta v^{(0)})$
%
  \begin{align*}
    \begin{bmatrix} \Delta x^{(0)} \\ \Delta v^{(0)} \end{bmatrix} \leftarrow 
    \begin{bmatrix} x^{\star(0)} \\ v^{\star(0)} \end{bmatrix}  - 
    \begin{bmatrix} x^{(0)} \\ v^{(0)} \end{bmatrix}.
  \end{align*}
  %
  In particular for $\Delta v^{(0)}$ we have
  %
  \begin{align*}
    \Delta v_i^{(0)} \leftarrow \left\{ 
    \begin{array}{l l}
      v_i^{\star(0)} - v_i^{(0)} = (a_i^Tx^{\star(0)} - b_i) - (a_i^Tx^{(0)} - b_i) = a_i^T\Delta x & \quad \text{if} \,\,\,\, i\in\mathcal{W}^{(0)} \\
      v_i^{\star(0)} - v_i^{(0)} = -v_i^{(0)} & \quad \text{if} \,\,\,\, i\not\in\mathcal{W}^{(0)}
    \end{array} \right.
  \end{align*}
  %
  where $b_i = \ubar{b}_i$ if \UseVerb{CTR_ACTIVE_LB} and $b_i = \bar{b}_i$ if \UseVerb{CTR_ACTIVE_UB} or \UseVerb{CTR_ACTIVE_EQ}.

\item The active-set search starts ...
\end{itemize}

\subsection{Initialize $v_i^{(0)}, i\in\mathcal{W}^{(0)}$}

\begin{align} \label{eq.v0_active}
  v_i^{(0)} \leftarrow \left\{ 
  \begin{array}{l l}
    a_i^Tx^{(0)} - \bar{b}_i & \quad \UseVerb{CTR_ACTIVE_UB} \,\,\,\, \text{\bf or} \,\,\,\, \UseVerb{CTR_ACTIVE_EQ} \\
    a_i^Tx^{(0)} - \ubar{b}_i & \quad \UseVerb{CTR_ACTIVE_LB}
  \end{array} \right.
\end{align}

\subsection{Initialize $v_i^{(0)}, i\not\in\mathcal{W}^{(0)}$} \label{sec.v0_inactive}
%
If $a_i^Tx^{(0)} \in [\ubar{b}_i,\bar{b}_i]$, $v_i^{(0)} \leftarrow 0$. Otherwise
%
\begin{align} \label{eq.v0_inactive}
  v_i^{(0)} \leftarrow \left\{ 
  \begin{array}{l l}
    a^T_ix^{(0)} - b_{\mathit{closest}} & \quad \UseVerb{set_min_init_ctr_violation} = \UseVerb{true} \\
    a^T_ix^{(0)} - \frac{1}{2}(\ubar{b}_i + \bar{b}_i) & \quad \UseVerb{set_min_init_ctr_violation} = \UseVerb{false}
  \end{array} \right.,
\end{align}
%
where $b_{\mathit{closest}}$ is the bound closest to $a^T_ix^{(0)}$. When
\UseVerb{set_min_init_ctr_violation} = \UseVerb{false} and one of $\abs{\bar{b}_i}$ or
$\abs{\ubar{b}_i}$ is very large, we might want to have an additional option. % todo

\clearpage

\subsection{Phase1 (alternative)} \label{sec.phase1_v0}

Instead of solving a \UseVerb{lexlse} problem during the initialization we could simply make a step
to reduce the constraint violations (for the inactive constraints) and keep $x^{(0)}$ constant. If a
blocking constraint is reached, it can added to the working set and only then (during the second
iteration) a \UseVerb{lexlse} problem must solved. Of course, if a full step can be taken
(\emph{i.e.,} $\alpha = 1$) we would have to solve a \UseVerb{lexlse} with $\mathcal{W}^{(1)} =
\mathcal{W}^{(0)}$. \UseVerb{phase1_v0} can be used by setting {\color{blue}$\UseVerb{use_phase1_v0}
  = \UseVerb{true}$}.
%
\begin{itemize}
\item Form $\mathcal{W}^{(0)}$ using Algorithm~\ref{alg.W0} (this could possibly modify $x^{\mathit{guess}}$).

\item Set $\Delta x^{(0)} = 0$

\item Set
  %
  \begin{align*}
    \Delta v_i^{(0)} \leftarrow \left\{ 
    \begin{array}{l l}
      0 & \quad \text{if} \,\,\,\, i\in\mathcal{W}^{(0)} \\
      -v_i^{(0)} & \quad \text{if} \,\,\,\, i\not\in\mathcal{W}^{(0)}
    \end{array} \right.
  \end{align*}
\end{itemize}

\subsubsection{Note for developers}

When \UseVerb{phase1_v0} is used, the behavior of the function \UseVerb{verifyWorkingSet} is
slightly modified (only during the first call and in the case when we can take a full step
\emph{i.e.,} $\alpha = 1$) - note the variable \UseVerb{normalIteration}. When using
\UseVerb{phase1_v0}, during the first call of \UseVerb{verifyWorkingSet} Lagrange multipliers are
not computed even if a full step could be taken (\emph{i.e.,} $\alpha = 1$). Hence, only two steps
are performed:
%
\begin{itemize}
\item Check for blocking constraints \emph{i.e.,} determine the largest $\alpha$ for which
  %
  \begin{align*}
  (x^{(0)}, v^{(0)} + \alpha \Delta v^{(0)})
  \end{align*}
  %
  is still a feasible pair.

\item Update $v^{(0)}$, \emph{i.e.,} take a step
  %
  \begin{align*}
    \begin{bmatrix} x^{(0)} \\ v^{(0)} \end{bmatrix} \leftarrow 
    \begin{bmatrix} x^{(0)} \\ v^{(0)} \end{bmatrix} +
    \alpha\begin{bmatrix} 0 \\ \Delta v^{(0)} \end{bmatrix}.
  \end{align*}
\end{itemize}
%
The active-set method starts with the next call of \UseVerb{verifyWorkingSet}.

\clearpage

\section{Algorithms}

\subsection{Determine the type of the $i$-th constraint}

\begin{algorithm} \label{alg.type}
  \caption{determine the type of the $i$-th constraint}

  \KwIn{\UseVerb{determine_ctr_type}($i$)}
  \KwData{\UseVerb{type}$\in\{$\UseVerb{CTR_INACTIVE}, \UseVerb{CTR_ACTIVE_LB}, \UseVerb{CTR_ACTIVE_UB}, \UseVerb{CTR_ACTIVE_EQ}$\}$ \\
    \hspace{1.4cm}(note: {\color{blue}\UseVerb{type} is the input from the user}), \\
    \hspace{1.4cm}\UseVerb{x_guess_is_specified}, \\
    \hspace{1.4cm}\UseVerb{modify_type_inactive_enabled}, \\
    \hspace{1.4cm}\UseVerb{modify_type_active_enabled}
  }
  \KwOut{\UseVerb{type}}
  
  \vspace{0.2cm}

  \If{$\ubar{b}_i = \bar{b}_i$}{\KwRet{\upshape\color{blue}\UseVerb{CTR_ACTIVE_EQ}}}

  \vspace{0.2cm}

  \If{\upshape\UseVerb{type} = \UseVerb{CTR_ACTIVE_EQ}}
      {
        \UseVerb{type} $\leftarrow$ \UseVerb{CTR_INACTIVE} \tcc{(possibly) report an error} 
      }

  \vspace{0.2cm}

  \If{\upshape\color{blue}\UseVerb{x_guess_is_specified}}
      {
        \vspace{0.2cm}
        
        \If{\upshape{\color{blue}\UseVerb{type} = \UseVerb{CTR_INACTIVE}} {\bf and} \color{blue}\UseVerb{modify_type_inactive_enabled}}{
          \If{\upshape$a_i^Tx^{\mathit{guess}} \leq \ubar{b}_i$}{\KwRet \UseVerb{CTR_ACTIVE_LB}}
          \If{\upshape$a_i^Tx^{\mathit{guess}} \geq \bar{b}_i$}{\KwRet \UseVerb{CTR_ACTIVE_UB}}
        }
        
        \vspace{0.2cm}
        
        \If{\upshape{\color{blue}\UseVerb{type} = \UseVerb{CTR_ACTIVE_LB}} {\bf and} \color{blue}\UseVerb{modify_type_active_enabled}}
           {
             \If{\upshape$a_i^Tx^{\mathit{guess}} > \ubar{b}_i$}
                {
                  \eIf{\upshape$a_i^Tx^{\mathit{guess}} < \bar{b}_i$}{\KwRet\UseVerb{CTR_INACTIVE}}{\KwRet\UseVerb{CTR_ACTIVE_UB}}
                }
           }
        
        \vspace{0.2cm}
        
        \If{\upshape{\color{blue}\UseVerb{type} = \UseVerb{CTR_ACTIVE_UB}} {\bf and} \color{blue}\UseVerb{modify_type_active_enabled}}
           {
             \If{\upshape$a_i^Tx^{\mathit{guess}} < \bar{b}_i$}
                {
                  \eIf{\upshape$a_i^Tx^{\mathit{guess}} > \ubar{b}_i$}{\KwRet\UseVerb{CTR_INACTIVE}}{\KwRet\UseVerb{CTR_ACTIVE_LB}}
                }
           }
      }
      \KwRet \UseVerb{type}
\end{algorithm}

\clearpage

\subsection{Update $x^{\mathit{guess}}$}

\begin{algorithm} \label{alg.x_guess}
  \caption{update $x^{\mathit{guess}}$}

  \KwIn{\UseVerb{update_x_guess}($x^{\mathit{guess}}$)}
  \KwData{\UseVerb{type}$\in\{$\UseVerb{CTR_INACTIVE}, \UseVerb{CTR_ACTIVE_LB}, \UseVerb{CTR_ACTIVE_UB}, \UseVerb{CTR_ACTIVE_EQ}$\}$ \\
    \hspace{1.4cm}(note: {\color{blue}\UseVerb{type} is the the output of Algorithm~\ref{alg.type}}), \\
    \hspace{1.2cm} \UseVerb{modify_x_guess_enabled}}
  \KwOut{(possibly) modified $x^{\mathit{guess}}$}
  
  \vspace{0.2cm}

  \eIf{\upshape\color{blue}\UseVerb{modify_x_guess_enabled} = \UseVerb{false}}{\KwRet $x^{\mathit{guess}}$}
      {
        \vspace{0.2cm}

        \For{$i \in \{$all constraints in first objective$\}$}
            {
              \vspace{0.2cm}
              
              $k \leftarrow $ index of element of $x$ participating in the $i$-th simple bound

              \vspace{0.2cm}

              \If{\upshape{\color{blue}\UseVerb{type} = \UseVerb{CTR_ACTIVE_LB}}}
                 {$x_k^{\mathit{guess}} \leftarrow \ubar{b}_i$}
           
              \vspace{0.2cm}
           
              \If{\upshape{\color{blue}\UseVerb{type} = \UseVerb{CTR_ACTIVE_UB}} {\bf or} {\color{blue}\UseVerb{type} = \UseVerb{CTR_ACTIVE_EQ}}}
                 {$x_k^{\mathit{guess}} \leftarrow \bar{b}_i$}
              
              \vspace{0.2cm}
              
              \If{\upshape{\color{blue}\UseVerb{type} = \UseVerb{CTR_INACTIVE}}}
                 {$x_k^{\mathit{guess}} \leftarrow \frac{1}{2}(\ubar{b}_i + \bar{b}_i)$ \tcc{other choices are possible (skip for now)}}
            }
      }
\end{algorithm}

\subsection{Form $W^{(0)}$}

\begin{algorithm} \label{alg.W0}
  \caption{form $W^{(0)}$}
  \KwData{\UseVerb{x_guess_is_specified}, \UseVerb{modify_x_guess_enabled}}
  
  \vspace{0.2cm}

  \eIf{\upshape{\color{blue}\UseVerb{x_guess_is_specified}} {\bf and} \\ 
               \hspace{0.45cm}{\color{blue}\UseVerb{modify_x_guess_enabled}} {\bf and} \\ 
               \hspace{0.45cm}{\color{blue}\UseVerb{SIMPLE_BOUNDS_OBJECTIVE}}}
      {
        \For{$i \in \{$all constraints in first objective$\}$}
            {
              \UseVerb{type}($i$) $\leftarrow$ \UseVerb{determine_ctr_type}($i$) 
            }

        \vspace{0.2cm}

        {\color{blue}\UseVerb{update_x_guess}($x^{\mathit{guess}}$)}
        
        \vspace{0.2cm}
        
        \For{$i \in \{$all constraints in remaining objectives$\}$}
            {
              \UseVerb{type}($i$) $\leftarrow$ \UseVerb{determine_ctr_type}($i$) 
            }
      }
      {
        \For{$i \in \{$all constraints$\}$}
            {
              \UseVerb{type}($i$) $\leftarrow$ \UseVerb{determine_ctr_type}($i$) 
            }
      }
\end{algorithm}

\clearpage

\section{Advanced initialization} \label{sec.advanced_initialization}

The user can initialize directly $\{\mathcal{W}^{(0)}, x^{(0)}, v^{(0)}\}$. This can be achieved by
specifying $v^{(0)}$ in addition to $\mathcal{W}^{\mathit{guess}}$ and $x^{\mathit{guess}}$. The bahevior of the solver then is 
%
\begin{enumerate}
\item Set $\mathcal{W}^{(0)} \leftarrow \mathcal{W}^{\mathit{guess}}$.
\item Detect equality constraints and add them to $\mathcal{W}^{(0)}$.
\item Set $x^{(0)} \leftarrow x^{\mathit{guess}}$.
\end{enumerate}
%
Several important notes:
%
\begin{itemize}
\item Hot-start with $\{\mathcal{W}^{(0)}, v^{(0)}\}$ or $\{v^{(0)}\}$ is not allowed.
\item If $(x^{(0)}, v^{(0)})$ is not a feasible initial pair, the behavior of the solver is
  unpredictable. Note that, no check of the feasibility of $(x^{(0)}, v^{(0)})$ is made.
\end{itemize}

\end{document}

                // --------------------------------------------------------
                // reduce constraint violation 
                // --------------------------------------------------------
                // We have two options:
                //
                // 1. Reduce constraint violation and aim at the (x_star,v_star).
                //    Hence, we need to solve a lexlse problem. 
                //
                // 2. Reduce constraint violation and simply add blocking 
                //    constraint (if such exists). Hence, we might be able to  
                //    make one step without solving a lexlse problem. Of course,
                //    it is not clear whether this is a step we want to make
                //    (todo: compare on some interesting problem)
                //   
                //    In order to implement the second option, maybe we can 
                //    add a call to checkBlockingConstraints(...) folowed by 
                //    a call to api_activate(...)
                //



%%%EOF
